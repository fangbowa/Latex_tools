\documentclass[12pt]{article}
%	options include 12pt or 11pt or 10pt
%	classes include article, report, book, letter, thesis

\usepackage{amsmath}

\usepackage{graphicx}
%\usepackage{graphics}
%\usepackage{subfigure}
\usepackage{bm}
\usepackage{subfig}
\usepackage{float}


\usepackage[usenames, dvipsnames]{color}
\usepackage{xcolor}

\usepackage{IEEEtrantools}

%\usepackage{titlesec}
%\titleformat{\section}{\normalfont\large\bfseries}{\thesection}{1em}{}
%\titleformat{\subsection}{\normalfont\normal\bfseries}{\thesubsection}{1em}{}
%\titlespacing\section{0pt}{12pt plus 4pt minus 2pt}{0pt plus 2pt minus 2pt}
%\titlespacing\subsection{0pt}{12pt plus 4pt minus 2pt}{0pt plus 2pt minus 2pt}

%to change the color of citation number, equation number
\usepackage[colorlinks=true,urlcolor=purple,citecolor=green,linkcolor=blue]{hyperref}

\usepackage{listings}
\lstset
{
    breaklines=true,
    basicstyle=\tt\scriptsize,
    keywordstyle=\color{magenta},
    identifierstyle=\color{blue},
    frame=single
}

\setlength{\parindent}{0pt}   %indent space
\setlength{\parskip}{1em}     %top, bottom space
\renewcommand{\baselinestretch}{1.0} % line spacing


\title{Fangbo's \LaTeX{} template}
\author{Fangbo Wang \\   }
\date{November 2017}

\begin{document}
\maketitle
\tableofcontents

\newpage
\section{Introduction}
aaaa
\section{Spacing and font size}

\subsection{For paragraph}
The most useful three commands
\begin{lstlisting}
\setlength{\parindent}{0pt}   %indent space
\setlength{\parskip}{1em}     %top, bottom space
\renewcommand{\baselinestretch}{1.0} % line spacing
\end{lstlisting}

\subsection{For heading}
use package \textit{titlesec} and commands \textit{titleformat}, \textit{titlespacing}
\begin{lstlisting}
\usepackage{titlesec}
\titleformat{\section}{\normalfont\large\bfseries}{\thesection}{1em}{}
\titleformat{\subsection}{\normalfont\normal\bfseries}{\thesubsection}{1em}{}
\titlespacing\section{0pt}{12pt plus 4pt minus 2pt}{0pt plus 2pt minus 2pt}
\titlespacing\subsection{0pt}{12pt plus 4pt minus 2pt}{0pt plus 2pt minus 2pt}
\end{lstlisting}


\section{Colored hyperlinks for figures, tables, citations}
use hyperref package to change the default color and set hyperlinks
%\begin{lstlisting}[language=LaTeX]
\begin{lstlisting}
\usepackage[colorlinks=true,urlcolor=purple,citecolor=green,linkcolor=blue]{hyperref}
\end{lstlisting}


\section{Citation}

1. write and cite a reference in this file itself. \cite{sett}
\begin{lstlisting}
\begin{thebibliography}{9}
\bibitem{sett} 
Arezoo Sadrinezhad, Kallol Sett and S. I. Hariharan. 
Efficient solution algorithms for multiaxial probablistic elasto-plastic constitutive simulations of soils.
\textit{Int J Numer Anal Mech Geomech}. 2017; 0:1-21.
\end{thebibliography}
\end{lstlisting}


\section{Display code}
In the preamble part, paste this
\begin{lstlisting}
\usepackage{listings}
\lstset
{
    breaklines=true,
    basicstyle=\tt\scriptsize,
    keywordstyle=\color{magenta},
    identifierstyle=\color{blue},
    frame=single
}
\end{lstlisting}

When pasting code in document, use begin\{lstlisting\} and end\{lstlisting\}

\begin{lstlisting}[language=Matlab]

clc; clear;
leftbound=-20; rightbound=20; meshpoint=2000;
stressmesh=linspace(leftbound, rightbound, meshpoint);
% standard normal assumption;
u00=exp(-stressmesh.*stressmesh/2)/sqrt(2*pi); 
timestart=10; dt=0.01;
N1=100;  N2=1;        

\end{lstlisting}



\section{Equations}

\subsection{one short equation (use \textit{equation})}
\begin{equation}
c=a+b
\end{equation}

\begin{lstlisting}
\begin{equation}
c=a+b
\end{equation}
\end{lstlisting}

\subsection{one long equation (use \textit{multiline})}
\begin{multline}
p(x) = 3x^6 + 14x^5y + 590x^4y^2 + 19x^3y^3 + 590x^4y^2 + 19x^3y^3\\ 
- 12x^2y^4 - 12xy^5 + 2y^6 - a^3b^3
\end{multline}

\begin{lstlisting}
\begin{multline}
p(x) = 3x^6 + 14x^5y + 590x^4y^2 + 19x^3y^3 + 590x^4y^2 + 19x^3y^3\\ 
- 12x^2y^4 - 12xy^5 + 2y^6 - a^3b^3
\end{multline}
\end{lstlisting}

\subsection{multiple equations (use \textit{eqnarray} ,\textit{align}, \textit{gather})}

\begin{align}
a & = b + c \\
& = d + e
\end{align}

\begin{lstlisting}
\begin{align}
a & = b + c \\
& = d + e
\end{align}
\end{lstlisting}

\begin{gather*} 
2x - 5y =  8 \\ 
3x^2 + 9y =  3a + c
\end{gather*}

\begin{lstlisting}
\begin{gather} 
2x - 5y =  8 \\ 
3x^2 + 9y =  3a + c
\end{gather}
\end{lstlisting}

\begin{eqnarray}
a & = & b + c \\
& = & d + e
\label{eq:faultyeqnarray}
\end{eqnarray}

\begin{lstlisting}
\begin{eqnarray}
a & = & b + c \\
& = & d + e
\end{eqnarray}
\end{lstlisting}

Note that \textit{align} only needs one \&, \textit{eqnarray} needs two \&.
Although the above commands work most of the time, it is recommmended to use \textit{IEEEeqnarray}. Read $<<$ How to typeset equations in Latex, Stefan Moser$>>$ for typing complex equations.

\begin{IEEEeqnarray}{rCl}
N_{(1)_m}^{q^{eq}} &=& P[f>0] N_{(1)_{m}}^{q^{ep}}  \\
N_{(2)_{mn}}^{q^{eq}} &=& P[f>0] N_{(2)_{mn}}^{q^{ep}}
\end{IEEEeqnarray}

\begin{lstlisting}
\begin{IEEEeqnarray}{rCl}
N_{(1)_m}^{q^{eq}} &=& P[f>0] N_{(1)_{m}}^{q^{ep}}  \\
N_{(2)_{mn}}^{q^{eq}} &=& P[f>0] N_{(2)_{mn}}^{q^{ep}}
\end{IEEEeqnarray}
\end{lstlisting}


\subsection{Collection of some equations}

\begin{eqnarray}
D_{ijkl} = 
\left\{
\begin{array}{ll}
a \\
b
\end{array} 
\right.
\end{eqnarray}

\begin{lstlisting}
\begin{eqnarray}
D_{ijkl} = 
\left\{
\begin{array}{ll}
a \\
b
\end{array} 
\right.
\end{eqnarray}
\end{lstlisting}


\begin{align}
x&=y           &  w &=z              &  a&=b+c\\
2x&=-y         &  3w&=\frac{1}{2}z   &  a&=b\\
-4 + 5x&=2+y   &  w+2&=-1+w          &  ab&=cb
\end{align}

\begin{lstlisting}
\begin{align}
x&=y           &  w &=z              &  a&=b+c\\
2x&=-y         &  3w&=\frac{1}{2}z   &  a&=b\\
-4 + 5x&=2+y   &  w+2&=-1+w          &  ab&=cb
\end{align}
\end{lstlisting}


\section{Making handouts for beamer presentations}

Ever wondered how to avoid the overlays when you want a printout from your beamer presentation?
Just put the handout option in your beamer presentation.
\begin{lstlisting}
\documentclass[12pt,handout]{beamer}
\end{lstlisting}

If you then want t o also print everything on one page (as Steve did for his Away Day talk) then just do this
\begin{lstlisting}[language=Tex]
\documentclass[a4paper]{article}
\usepackage{pdfpages}
 
\begin{document}
\includepdf[pages=1­-last,nup=2x2,landscape=false,frame=true,
            noautoscale=true,scale=0.6,delta=0mm 5mm]{mypresentation.pdf}
\end{document}
\end{lstlisting}
where mypresentation.pdf is your beamer presentation which you created with the handout options (so without overlays).
In this example 4 beamer slides are printed on one page (2x2).

\begin{thebibliography}{9}
\bibitem{sett} 
Arezoo Sadrinezhad, Kallol Sett and S. I. Hariharan. 
Efficient solution algorithms for multiaxial probablistic elasto-plastic constitutive simulations of soils.
\textit{Int J Numer Anal Mech Geomech}. 
2017; 0:1-21.
 
\bibitem{Fu} 
Chuli Fu, Xiangtuan Xiong and Zhi Qian. 
Fourier regularization for a backward heat equation.
\textit{Jounal of Mathematical Analysis and Applications}. 
2007; 331:471-480.


\end{thebibliography}


\end{document}
